\chapter{Introduction}

Web application vulnerabilities like cross site scripting (XSS) and SQL injection attacks have devastating effect in the Internet. Though these vulnerabilities are well known and  preventive measures are well documented a lot of application continue to suffer from such attacks. 

Similarly a growing number of web applications use other thrid-party web applications directly on their web pages in a technique known as mash-up. It is very common for many websites to embed Google or other mapping gadgets to display directions or URL bookmarking and sharing widgets. Also a many publication platforms, news outlets and ecommerce sites display advertisements but don't control either their origin or content.

It is highly desireable to restrict access to the origin page's resources from these third-party scripts and seperate application security from application functioanlity through languge runime support combined with seemless application framework integration. For example majory of the modern languages like Java, JavaScript, Go, etc., provide built in garbage collection, safe memory and array access gurantees. On the same vein we explore the approach of dynamic taint detection to ensure integrity of the data and dynamic information flow to gurantee confidentiality. We lean towards dynamic checking as opposed of static checking  [meyes 1999; meyers liskov] 1997 to provide better user experience by avoiding expensinve upfront static anlysis on each client browser before execution. Dynamic analysis also allows for more flexibility in tailoring policies based on browser capabilites and hot swap policies [chandra and franz 2007].

We achive information flow tracking and taint detection using universal labeling where every value has an associated information flow label these labels makes tracking easy and staright forward. However labeling every value in the program and propagating it as the values are modified has significant overhead so we leave it up to the frameworks to label any suspect values in case taint detection and to the programmers to label any sensitive data whose flow they want to track and restrict.

Applications perform any given  task by executing operaions on data  by virtualizing the interface between operations and data Sweet.JS produce virtual values. Any operation on values that are proxied by such virtual values invokes a trap on the vitual value. Each virtual value is just a collection of such traps corresponding to various legal operations that can be performed on the underlying value. These traps are inturn user defined functions which describe how a specific operation should be behave. We use virtual values to label the values.

Languages like JavaScript, Java and SmallTalk, provide facility to proxy operations on any object values either to override the behavior or delegate to the underlying object this is known as \textit{behavioral intercession}. While all values in SmallTalk are objects values in JavaScript or Java can be either a primitive or an object and there are no language level feature to proxy primitive values to achieve behavirol intercession. Sweet.JS virtual values can wrap any primitive value as easily as object values and make them amenable for behavioral intercession.

Aspect-oriented programming (AOP) is another approach that focuses on \textit{cross-cutting concerns} that span across various modules. These systems allow developers to define \textit{point-cuts} that are instrumented to provider custom behavior.  While virtual values allow developers to instrument code to modify the behavior akin to AOP it is rather done dynamically at the runtime  as oppsed to static weaving performed by AOP.  One such implementation [Object Views Leo  Mller] due to its static nature performs server-side rewiting to instrument the code this renders the resultant code very difficult to understand and debug.

This paper is organized as follows. This next chapter procvides background information on web application vulnerabilities, taint and Information flow analysis and macro systems. in chapters 3 and 4 we provide implementaion details of taint analysis and information flow analysis. Chapter 6 discusses realated work and future improvements to the implementation.
